%% Time-stamp: <2019-01-22 14:47:04 (marc)>
\documentclass[xcolor=x11names,compress, mathserif]{beamer}
\newcommand\hmmax{0}
\newcommand\bmmax{0}

\usepackage{../includes/MarkMathCmds}

\newcommand{\hackspace}{\hspace{4.2mm}}
\newcommand{\showstudent}[1]{}



% talk/author information
\newcommand{\authorname}{Mark van der Wilk}
\newcommand{\authoremail}{m.vdwilk@imperial.ac.uk}
\newcommand{\authoraffiliation}{
 Department of Computing\\Imperial
  College London}
\newcommand{\authortwitter}{markvanderwilk}
\newcommand{\slidesettitle}{\imperialBlue{Decision Theory}}
\newcommand{\footertitle}{Decision Theory}
\newcommand{\location}{Imperial College London}
\newcommand{\talkDate}{{February 13, 2023}}




\date{\imperialGray{\talkDate}}




% load defaults
\input{../includes/header.tex}
\input{../includes/titlepage.tex}



\begin{frame}{Taking Action}
What do you want to \emph{use} a posterior for?
\begin{itemize} \pause
\item Posterior is a belief over what will happen. \pause
\item You use beliefs to take action / make decisions. \pause
\item Uncertainty helps you to balance risk, reward, and new information. \pause
\end{itemize}
\begin{center}
\large We \emph{learn} about the world, \\ so we can \emph{act} in it, \\ to get outcomes that we \emph{desire}.
\end{center}
\end{frame}


\begin{frame}{Axiomatisation}



As with uncertain reasoning, we can get maths to tell us how to make rational decisions, if we write down our wishlist of axioms. \pause
\\
Paraphrased axioms: \pause
\begin{enumerate}
\item Comparing probability distributions A and B over outcomes, you either prefer one over the other, or you have no preference. \pause
\item If you prefer A over B, and B over C, then you prefer A over C. \pause
\item Reduction of compound lotteries: If you prefer A over B, then you also prefer a 50\% chance of getting A over a 50\% chance of getting B, with a 50\% chance of getting C. \pause
\end{enumerate}


{\tiny See Ch 16 ``Artificial Intelligence: A Modern Approach'', Russell \cite{russell}}
\end{frame}


\begin{frame}{Von Neumann-Morgenstern Utility Theorem}
\begin{itemize}
\item If you behave according to these axioms, then there exists a \emph{utility function} that you maximise. \pause
\item In other words, if you believe you want to act according to those axioms, you should follow the principle of \emph{Maximum Expected Utility}. \pause
\end{itemize}
\end{frame}

\begin{frame}
\begin{myblock}{Principle of Maximum Expected Utility}
\begin{enumerate}
\item Define a utility function $U: \mathcal X \times \mathcal A \to \Reals$ \\
$\mathcal X$: Space of outcomes. $\mathcal A$: Space of actions.
Quantifies how good an outcome is, if you take a particular action. \pause
\item Compute expected utility for your actions. \\
Your beliefs are a distribution over outcomes given action.
\begin{align}
u(a) = \Exp{p(x|\mathcal D, a)}{U(x, a)}
\end{align} \pause
\item At the time of decision making, choose action which maximises expected utility.
\begin{align}
a^* = \argmax_a u(a)
\end{align}
\end{enumerate}
\end{myblock}
\end{frame}

\begin{frame}
\begin{myblock}{Exercise: Why do we predict the mean?}
Your regression model gives you $p(y|X,\vy)$. You need to give a ``best guess'' $y_p$, and your utility will be $U(y, y_p) = -(y-y_p)^2$. \\
\arrow Find $y_p$ in terms of properties of $p(y|X,\vy)$.
\end{myblock} \pause
Solution: We want $\deriv[]{y_p} u(y_p) = \deriv[]{y_p} \Exp{p(y|X,\vy)}{U(y, y_p)} = 0$. \pause
\begin{align*}
\deriv[]{y_p}u(y_p) &= \deriv[]{y_p} \int_{-\infty}^{\infty} p(y|X,\vy) U(y, y_p) \calcd y \Pause \\
&= -\int p(y|X,\vy) \pderiv[]{y_p} (y-y_p)^2 \calcd y  && \text{\tiny Leibniz Integral Rule} \\ \Pause
&= 2\int p(y|X,\vy) (y-y_p) \calcd y \\ \Pause
0 &= 2\Exp{p(y|X,\vy)}{y} - 2y_p \\ \Pause
y_p &= \Exp{p(y|X,\vy)}{y}\,, \qquad \text{i.e. } y_p \text{ should be the \emph{mean}!}
\end{align*}
\end{frame}



\begin{frame}
\vspace{-0.3cm}
\begin{myblock}{Exercise: Absolute error loss.}
Your regression model gives you $p(y|X,\vy)$. You need to give a ``best guess'' $y_p$, and your utility will be $U(y, y_p) = -|y-y_p|$. \\
\arrow Find $y_p$ in terms of properties of $p(y|X,\vy)$.
\end{myblock} \pause
Solution: We want $\deriv[]{y_p} u(y_p) = \deriv[]{y_p} \Exp{p(y|X,\vy)}{U(y, y_p)} = 0$. \pause
\begin{align*}
\deriv[]{y_p}u(y_p) &= \deriv[]{y_p} \int_{-\infty}^{\infty} p(y|X,\vy) U(y, y_p) \calcd y \Pause \\
&= \int p(y|X,\vy) \pderiv[]{y_p} U(y, y_p) \calcd y  && \text{\tiny Leibniz Integral Rule} \\ \Pause
&= \int p(y|X,\vy) \begin{rcases}\begin{dcases}-1 & \text{if } y > y_p \\ 1 & \text{if } y < y_p  \end{dcases}\end{rcases}\calcd y \Pause \\
&= -\int_{-\infty}^{y_p} p(y|X,\vy) \calcd y + \int_{y_p}^{\infty} p(y|X,\vy) \calcd y \Pause \\
0 &= -F(y_p) + (1-F(y_p)) && \text{$F$ is cdf} \Pause \\
\implies & y_p = F^{-1}(0.5)\,, \qquad \text{i.e. } y_p \text{ should be the \emph{median}!}
\end{align*}
\end{frame}


\begin{frame}
Reject option (todo)
\end{frame}


\begin{frame}{Applications and Applicability}
Hugely influential theory:
\begin{itemize}
\item Philosophy: Utilitarianism {\tiny effective altruism}
\item Psychology: Rational behaviour as a model for human behaviour.
\item Economics: How to optimise investments.
\item Game theory: Analyse implications of rational choice.
\item Politics: Voting systems (Arrow's impossibility theorems).
\end{itemize}

\pause
Discussions about desirability/realism of axioms:
\begin{itemize}
\item Naïve application leads to bad behaviour (St Petersburg Paradox)
\item Can you express desires as utility functions?
\item The Law of Perverse Optimization: Whenever a desired behaviour is formulated as a utility, optimising for the utility will give you behaviour you didn't want. (AI Paperclip factory)
\item Human's don't behave according to the axioms. But perhaps for good reason?
\item Bounded rationality: It assumes you can compute the optimal decision.
\end{itemize}
\end{frame}


\begin{frame}{Conclusion}
\begin{itemize}
\item Decision theory is easy (in principle): See slide 5. \pause
\item The implications of decision theory are vast, but sadly we don't have time for this. \pause
\item Next: Things that are easy in principle, are often very hard in practice. \pause
\end{itemize}

Recommended reference: MacKay \cite{itila} chapter 36.

Further reading: Russell \cite{russell} chapter 16.
\end{frame}





%%%%%%%%%%%%%%%%%%%%%%%%%%%%%%%%%%%%%%%%%
% REFERENCES
%%%%%%%%%%%%%%%%%%%%%%%%%%%%%%%%%%%%%%%%%
\begin{frame}[t,allowframebreaks]
\frametitle{References}
\linespread{1.0}
\tiny
\bibliographystyle{abbrv}
\bibliography{../includes/pi-literature.bib}
\end{frame}



\end{document}
%%% Local Variables: 
%%% mode: latex
%%% TeX-master: t
%%% End: 
