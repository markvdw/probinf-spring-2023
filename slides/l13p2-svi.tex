%% Time-stamp: <2019-03-01 13:49:57 (marc)>
\documentclass[xcolor=x11names,compress, mathserif,xcolor=table]{beamer}
\newcommand\hmmax{0}
\newcommand\bmmax{0}

\newcommand{\hackspace}{\hspace{4.2mm}}
\newcommand{\showstudent}[1]{}





% talk/author information
\newcommand{\authorname}{Mark van der Wilk}
\newcommand{\authoremail}{m.vdwilk@imperial.ac.uk}
\newcommand{\authoraffiliation}{
 Department of Computing\\Imperial
  College London}
\newcommand{\authortwitter}{markvanderwilk}
\newcommand{\slidesettitle}{\imperialBlue{Stochastic Variational Inference}}
\newcommand{\footertitle}{Stochastic Variational Inference}
\newcommand{\location}{Imperial College}
\newcommand{\talkDate}{February 25, 2022}

\newcommand{\lb}{\mathcal{L}}



\date{\imperialGray{\talkDate}}

\usepackage{../includes/MarkMathCmds}



% load defaults
\input{../includes/header.tex}
\input{../includes/titlepage.tex}

\linespread{1.2}


\begin{frame}{Recap: Variational Inference}
\begin{itemize}
\item KL measures discrepancy between distributions
\begin{align}
\KL{q(\vz)}{p(\vz\given\vx)} \geq 0 && \text{with equality iff } q(\vz) = p(\vz\given\vx)
\end{align}
\item Find approx $q_\vv(\vz) \approx p(\vz\given\vx)$ by minimising KL divergence:
\begin{align}
\vv^* = \argmin_{\vv} \KL{q_\vv(\vz)}{p(\vz\given\vx)}
\end{align}
\item Equivalent to maximising lower bound (ELBO) $\lb$ since
\begin{gather}
\KL{q_\vv(\vz)}{p(\vz\given\vx)} = \log p(\vx) - \lb(\vv) \\
\implies \vv^* = \argmax_\vv \lb(\vv)
\end{gather}
\end{itemize}
\end{frame}



\begin{frame}{VI for Conditionally Conjugate Models}
  For the class of \emph{conditionally conjugate models}, i.e.~models with complete conditionals in exponential family (e.g.~Bernoulli, Beta, Gamma, Gaussian, ...) and \emph{mean-field} (independent) variational approximations.
\begin{center}
  \begin{tikzpicture}
    \node[latent](z) at (-2,0) {$\vec z_n$};
    \node[obs](x) at (0,0) {$\vec x_n$};
    \node[latent](beta) at (-1,1){$\vec\beta$};
    %\edge{z,mu,sigma}{x};
    \edge{z}{x}
    \edge{beta}{x,z}
    \plate {xz} {(x)(z)} {$n=1,\dotsc, N$} ;
    \node[const] at (-4,0) {\small Local variables};
    \node[const] at (-3.9,1) {\small Global variables};
\end{tikzpicture}
%%% Local Variables:
%%% mode: latex
%%% TeX-master: "../lecture_variational_inference"
%%% End:

\end{center}


  \begin{itemize}
    \item We have \emph{closed-form} expression for ELBO
    \item Coordinate-ascent algorithm for maximising ELBO
    \item Important if you want to be a VI researcher, but not enough time.
  \end{itemize}
\end{frame}


\begin{frame}{Overview of today}
\begin{itemize}
  \item Limitations of Conditionally-Conjugate VI
  \item Black-box variational inference
  \item Gradients of expectations
\end{itemize}
\end{frame}




\begin{frame}
\begin{center}
\Large \emph{Limitation 1: Non-conjugate models}
\end{center}
\end{frame}




%%%%%%%%%%%%%%%%%%%%%%%%%%%%%%%%%%%%%%
\begin{frame}{Example: Bayesian Logistic Regression}
  \begin{columns}
\column{0.7\hsize}
    \begin{itemize}
    \item Binary classification
    \item Inputs $x \in\R$, labels $y\in\{0,1\}$
    \item Model parameter $z$ (normally denoted by $\theta$)
    \end{itemize}
        \column{0.3\hsize}

    \begin{center}
    \includegraphics[height=3cm]{./figures/vi/logistic_regression_prediction}
  \end{center}
\end{columns}
\pause
\begin{align*}
    &\text{Prior on model parameter: }p(z) = \gauss{0}{1}\\
    &\text{Likelihood: } p(y_n|x_n, z) = \Ber(\sigma(zx_n))
  \end{align*}
  \vspace{-5mm}
  \pause
  \begin{itemize}
  \item Assume we have a single data point $(x,y)$
  \item Goal: Approximate the intractable posterior distribution $p(z|x,y)$
    using variational inference
  \end{itemize}
\end{frame}

%%%%%%%%%%%%%%%%%%%%%%%%%%%%%%%%%%%%%%
\begin{frame}{Example: Bayesian Logistic Regression (2)}
  \begin{itemize}
  \item Choose Gaussian variational approximation: \\
    $q_\vv(z) = \gauss{z; \mu}{\sigma^2}$ \arrow
    $\vec\nu = \onslide+<2->{\{\mu, \sigma^2\}}$
\onslide+<3->{
  \item Objective function: ELBO $\mathcal F(\vec\nu)$
  \begin{align*}
    \mathcal F(m,\sigma^2) &= \E_q[\orange{\log p(z)}  -\blue{\log 
                             q(z)} + \red{\log
                              p(y|x,z)}]\\
    \onslide+<4->{
                    &  = \orange{-\tfrac{1}{2}(\mu^2 +
                             \sigma^2)} +
                             \blue{\tfrac{1}{2}\log\sigma^2} +  \red{\E_q[\log
                             p(y|x,z)]}  +\text{c}}\\
   \onslide+<5->{\red{\E_q[\log p(y|x,z)]}&=}
\onslide+<6->{\E_q[y\log\sigma(xz)+(1-y)\log(1-\sigma(xz))]}
    \\
    \onslide+<7->{&= \E_q[yxz] - \E_q[y\log(1+\exp(xz))] \\
                         & \quad+
                             \E_q\left[(1-y)\log \left(1 -
                           \frac{\exp(xz)}{1+\exp(xz)}\right)\right]
                           }
  \end{align*}
\onslide+<7->{
  with
  \begin{align*}
    \sigma(xz) &= \frac{\exp(xz)}{1+\exp(xz)}
  \end{align*}
  }
}
\end{itemize}
  
\end{frame}



\begin{frame}{Computing the Expected Log-Likelihood}

  \begin{align*}
  \red{\E_q[\log p(y|x,z)]}&= \E_q[yxz] - \E_q[y\log(1+\exp(xz))] \\
                           &\quad +
                             \E_q[(1-y)\log \left(1 -
                             \frac{\exp(xz)}{1+\exp(xz)}\right)]\\
   \onslide+<2->{ &= yx\mu - \E_q[y\log(1+\exp(xz))]\\&\quad +
                             \E_q[(1-y)\log \left(
    \frac{1}{1+\exp(xz)}\right)]\\}
\onslide+<3->{    &= yx\mu -\E_q[y\log(1+\exp(xz))] \\
                          &\quad
                            - \E_q[\log (1 + \exp(xz))]  + \E_q[y\log
                            (1+ \exp(xz))]\\}
\onslide+<4->{    &=\red{yx\mu - \E_q[\log (1 + \exp(xz))] }}
  \end{align*}
\end{frame}



%%%%%%%%%%%%%%%%%%%%%%%%%%%%%%%%%%%%%%
\begin{frame}{Example: Bayesian Logistic Regression (ctd.)}
  \begin{itemize}
  \item Choose Gaussian variational approximation: \\
    $q_\vv(z) = \gauss{z; \mu}{\sigma^2}$ \arrow
    $\vec\nu = \{\mu, \sigma^2\}$
  \item Objective function: ELBO $\mathcal F(\vec\nu)$
  \end{itemize}
  \begin{align*}
    \mathcal F(\mu,\sigma^2) &= \E_q[\orange{\log p(z)} + \red{\log
                             p(y|x,z)} -\blue{\log 
                             q(z)}]\\
                           &= \orange{-\tfrac{1}{2}(\mu^2 +
                             \sigma^2)} +
                             \blue{\tfrac{1}{2}\log\sigma^2} +  \red{\E_q[\log
                             p(y|x,z)]}  +\text{c}\\
                           &= \orange{-\tfrac{1}{2}(\mu^2 +
                             \sigma^2)} +
                             \blue{\tfrac{1}{2}\log\sigma^2} +
                             \red{yx\mu - \E_q[\log (1 + \exp(xz))] } 
  \end{align*}
  \vspace{-5mm}
  \pause
  \begin{itemize}
  \item \alert{Expectation cannot be computed in closed form}
    % \item \calert{Pushing gradients through Monte Carlo estimates is very hard.}
  \item We want to optimise w.r.t.~variational parameters $\mu, \sigma^2$.
    \pause
  \item How can we optimise quantities that we cannot compute in closed-form?
  \end{itemize}
  
\end{frame}




\begin{frame}{Non-Conjugate Models}

    \begin{itemize}
    \item Nonlinear time series models
    \item Deep latent Gaussian models
    \item Attention models (e.g., DRAW)
    \item Generalized linear models (e.g., logistic regression)
    \item Bayesian neural networks
    \item ...
    \end{itemize}
\pause
    There are many interesting non-conjugate models
    \\\arrow Look for a solution that is not model specific\\
    \arrow \emph{Black-Box Variational Inference}
\end{frame}






\begin{frame}
\begin{center}
\Large \emph{Limitation 2: Large datasets}
\end{center}
\end{frame}


\begin{frame}{Example: Bayesian Logistic Regression}
Usual formulation:
\begin{align}
p(y_n\given \vx_n, \vz) = \mathrm{Ber}(\sigma(\vtheta\transpose\vx_n)) \\
p(\vtheta) = \NormDist{\vtheta; 0, \mathbf{I}}
\end{align}

ELBO:
\begin{align}
\lb &= \Exp{q(\vtheta)}{\log\prod_{n=1}^Np(y_n\given\vx_n,\vtheta)} - \KL{q(\vtheta)}{p(\vtheta)} \nonumber \\
&= \sum_{n=1}^N \Exp{q(\vtheta)}{\log p(y_n\given\vx_n,\vtheta)} - \KL{q(\vtheta)}{p(\vtheta)}
\end{align}
\end{frame}


\begin{frame}{Big data}
\begin{align}
\lb &= \sum_{n=1}^N \Exp{q(\vtheta)}{\log p(y_n\given\vx_n,\vtheta)} - \KL{q(\vtheta)}{p(\vtheta)}
\end{align}

In ``big data'' applications, $N$ may be millions or billions.

\vspace{0.3cm}

\arrow Summing over all datapoints at \emph{each} optimisation iteration for $q(\vtheta)$ is \emph{too slow}.

\end{frame}




\begin{frame}
\begin{center}
\Large \emph{Stochastic Optimisation}
\end{center}
\end{frame}


\begin{frame}{Stochastic Optimisation}




\begin{align}
\lb &= \sum_{n=1}^N \Exp{q(\vtheta)}{\log p(y_n\given\vx_n,\vtheta)} - \KL{q(\vtheta)}{p(\vtheta)}
\end{align}


We can trivially find an \emph{unbiased estimator} of the ELBO and its gradient by subsampling the data points! (solves problem 2)

\pause

\begin{align}
\hat \lb = \frac{N}{M} \sum_{n\in \mathcal{M}} \Exp{q_\vv(\vtheta)}{\log p(y_n\given \vx_n, \vtheta)} - \KL{q_\vv(\vtheta)}{p(\vtheta)} \\
\hat{\pderiv[\lb]{\vv}} = \frac{N}{M} \sum_{n\in \mathcal{M}} \pderiv{\vv} \Exp{q_\vv(\vtheta)}{\log p(y_n\given \vx_n, \vtheta)} - \pderiv{\vv}\KL{q_\vv(\vtheta)}{p(\vtheta)}
\end{align}


\begin{center}
\Large Can we still optimise with estimated gradients? \pause (Yes)
\end{center}


\end{frame}


\begin{frame}{Stochastic Gradient Descent (MML / Comp Opt)}
Goal: $\vv^* = \argmax_\vv \lb(\vv)$
\pause

Normal gradient descent:
\begin{gather}
\vv_t = \vv_{t-1} + \rho_t\nabla_\vv\lb(\vv_{t-1}) \\
\vv_t \to \vv^* \text{ as } t \to \infty
\end{gather}
\pause

Stochastic gradient descent (Robbins \& Monro, 1951):
\begin{gather}
\text{if } \Exp{}{\hat{g}_t} = \nabla_\vv \lb(\vv_{t})
\end{gather}
\vspace{-0.6cm}
\begin{gather}
\vv_t = \vv_{t-1} + \rho_t\hat{g}_t \\
\vv_t \to \vv^* \text{ as } t \to \infty \qquad \qquad \text{if } \sum_{t=1}^\infty\rho_t = \infty \text{ and } \sum_{t=1}^\infty\rho_t^2 < \infty \\
\text{e.g. } \rho_t = 1 / t
\end{gather}

\pause

Having a small $\mathbb{V}\left[\hat g_t\right]$ is crucial to ensure fast convergence.
\end{frame}


\begin{frame}{Stochastic Optimisation}
\begin{itemize}
\item Stochastic optimisation solves problem 2.
\item Still stuck with problem 1: Intractable integrals in VI.
\end{itemize}

\vspace{0.5cm}

\pause

Since we're using stochastic gradient estimates anyway... \pause
\begin{center}
\Large Can we not also find Monte Carlo approximations to the gradients of intractable integrals?
\end{center}
\end{frame}





\begin{frame}
\begin{center}
\Large \emph{Black-Box Variational Inference (BBVI)}
\end{center}
\end{frame}


\begin{frame}{Black-Box Variational Inference}
  \begin{center}
    \includegraphics[width=0.8\hsize]{./figures/vi/bbvi}
  \end{center}
  \begin{itemize}
  \item Any model (limitation 1)
  \item Massive data (limitation 2)
  \item Some general assumptions on the approximating family
  \end{itemize}
\end{frame}



\begin{frame}{Black-Box Variational Inference}
Problem 1: Intractable integral of the expected log-likelihood term
\begin{align}
\Exp{q(\vz)}{\log p(\vx\given\vz)} \,.
\end{align}

For stochastic optimisation we need an estimator of its \emph{gradient} $\hat g_t$, such that
\begin{align}
\Exp{}{\hat g_t} = \nabla_\vv \lb(\vv)
\end{align}

\pause
\vspace{0.5cm}

\begin{center}
\Large Can we find such unbiased estimates?
\end{center}

\pause

\begin{itemize}
\item Score function estimator
\item Reparameterisation estimator
\end{itemize}
\end{frame}


\begin{frame}{Problem statement}
We have intractable terms that can be written as:
\begin{align}
\Exp{q_\vv(\vz)}{h(\vz, \vv)}
\end{align}

Goal: Find estimator $\hat g$ with property
\begin{align}
\Exp{}{\hat g} = \nabla_\vv \Exp{q_\vv(\vz)}{h(\vz, \vv)}
\end{align} \pause

Remember:
\begin{itemize}
\item It's easy to find a MC estimate of the objective.
\item But we need an MC estimate of the gradients!
\end{itemize}

\end{frame}


% \begin{frame}{Re-Writing the ELBO}
%   \begin{align*}
%     \text{ELBO} &= \E_q[\log p(\vec x|\vec z)] - \KL{q(\vec
%                   z|\vec\nu)}{p(\vec z)} \\
%   \onslide+<2->{
%     &=\E_q[\blue{\log p(\vec x|\vec z)}] + \E_q[\blue{\log p(\vec z)} - \log q(\vec
%     z|\vec \nu)]\\}
% \onslide+<3->{   & = \E_q[\underbrace{\blue{\log p(\vec x, \vec z)} - \log q(\vec
%      z|\vec\nu)}_{=:h(\vec z, \vec\nu)}]}
%   \end{align*}

%   \onslide+<4->{
%     \begin{myblock}{Evidence Lower Bound}
%       \vspace{-5mm}
% \begin{align*}
%   \text{ELBO}&=\E_q[h(\vec z,\vec\nu)] =  \int h(\vec z, \vec\nu) q(\vec
%                z|\vec\nu)d\vec z\\
%   h(\vec z,\vec\nu) &:= \log p(\vec x, \vec z) - \log
%                       q_\vv(\vec z)
% \end{align*}
% \end{myblock}
% }
% \end{frame}



\begin{frame}[t]{Approach}
\begin{align}
g(\vv) = \nabla_\vv\Exp{q}{h(\vz, \vv)}
\end{align}
% \begin{center}
%     \tikzstyle{ipe stylesheet} = [
  ipe import,
  even odd rule,
  line join=round,
  line cap=butt,
  ipe pen normal/.style={line width=0.4},
  ipe pen heavier/.style={line width=0.8},
  ipe pen fat/.style={line width=1.2},
  ipe pen ultrafat/.style={line width=2},
  ipe pen normal,
  ipe mark normal/.style={ipe mark scale=3},
  ipe mark large/.style={ipe mark scale=5},
  ipe mark small/.style={ipe mark scale=2},
  ipe mark tiny/.style={ipe mark scale=1.1},
  ipe mark normal,
  /pgf/arrow keys/.cd,
  ipe arrow normal/.style={scale=7},
  ipe arrow large/.style={scale=10},
  ipe arrow small/.style={scale=5},
  ipe arrow tiny/.style={scale=3},
  ipe arrow normal,
  /tikz/.cd,
  ipe arrows, % update arrows
  <->/.tip = ipe normal,
  ipe dash normal/.style={dash pattern=},
  ipe dash dashed/.style={dash pattern=on 4bp off 4bp},
  ipe dash dotted/.style={dash pattern=on 1bp off 3bp},
  ipe dash dash dotted/.style={dash pattern=on 4bp off 2bp on 1bp off 2bp},
  ipe dash dash dot dotted/.style={dash pattern=on 4bp off 2bp on 1bp off 2bp on 1bp off 2bp},
  ipe dash normal,
  ipe node/.append style={font=\normalsize},
  ipe stretch normal/.style={ipe node stretch=1},
  ipe stretch normal,
  ipe opacity 10/.style={opacity=0.1},
  ipe opacity 30/.style={opacity=0.3},
  ipe opacity 50/.style={opacity=0.5},
  ipe opacity 75/.style={opacity=0.75},
  ipe opacity opaque/.style={opacity=1},
  ipe opacity opaque,
]
\definecolor{red}{rgb}{1,0,0}%
\begin{tikzpicture}[ipe stylesheet]
  \draw[shift={(311.397, 686.186)}, xscale=0.2262, yscale=0.0959]
    (0, 0)
     .. controls (10.6667, 26.6667) and (58.6667, 74.6667) .. (109.3333, 69.3333)
     .. controls (160, 64) and (213.3333, 5.3333) .. (234.6667, -50.6667)
     .. controls (256, -106.6667) and (245.3333, -160) .. (208, -200)
     .. controls (170.6667, -240) and (106.6667, -266.6667) .. (88, -248)
     .. controls (69.3333, -229.3333) and (96, -165.3333) .. (96, -122.6667)
     .. controls (96, -80) and (69.3333, -58.6667) .. (42.6667, -45.3333)
     .. controls (16, -32) and (-10.6667, -26.6667) .. cycle;
  \pic
     at (339.3985, 680.6362) {ipe disk};
  \draw[shift={(339.398, 680.636)}, scale=0.1686, -{>[ipe arrow small]}]
    (0, 0)
     .. controls (48, 16) and (112, -48) .. (158.005, -9.169);
  \pic
     at (366.0336, 679.0905) {ipe disk};
  \node[ipe node]
     at (84, 708) {$p(\vec x, \vec z)$};
  \node[ipe node]
     at (84, 648) {$q(\vec z|\vec \nu)$};
  \draw[shift={(196, 696)}, xscale=1.25,  ipe pen fat]
    (0, 0) rectangle (64, -32);
  \node[ipe node]
     at (204.891, 677.511) {$\displaystyle \int (...) q(\vec z|\vec
       \nu) d\vec z$};
  \node[ipe node]
     at (141.371, 677.339) {$\nabla_{\vec \nu}$};
  \draw[ipe pen fat]
    (132, 696) rectangle (164, 664);
  \draw[->]
    (164, 680)
     -- (196, 680);
  \draw[->]
    (276, 680)
     -- (308, 680);
  \draw[->]
    (112, 656)
     -- (132, 664);
  \draw[->]
    (112, 704)
     -- (132, 696);
  \node[ipe node, font=\tiny]
     at (245, 646.47) {\it \sffamily Adopted from Blei et al.'s NIPS-2016 tutorial};
\end{tikzpicture}

%%% Local Variables:
%%% mode: latex
%%% TeX-master: "../lecture_variational_inference"
%%% End:

%   \end{center}
  \begin{itemize}[<+->]
    \item Switch order to integration first, then differentiation \\
(Monte Carlo estimates need expectations, and expectations are integrals)
    \item Write integration as expectation again
    \item Approximate the expectation after having taken the gradient\newline
      \arrow Monte Carlo estimator (ideally with low variance)
    \item Stochastic optimization
    \end{itemize}
    \onslide+<5->{
    \arrow Require: \cemph{general way to compute gradients of
      expectations}}
\end{frame}



\begin{frame}{Log-Derivative Trick}
  \begin{myblock2}{Log-Derivative Trick}
\vspace{-2mm}
    \begin{align*}
      &\nabla_{\vec\nu} \log q_\vv(\vec z) = 
      \onslide+<2->{\frac{\nabla_{\vec\nu}q_\vv(\vec z)}{q_\vv(\vec z)}\\
      \iff & \nabla_{\vec\nu}q_\vv(\vec z) = q_\vv(\vec
             z)\nabla_{\vec\nu} \log q_\vv(\vec z) }
    \end{align*}
  \end{myblock2}
  \onslide+<3->{
  \begin{itemize}
    \item 
  Therefore:
  \begin{align*}
    \int \nabla_{\vec\nu} q_\vv(\vec z) f(\vec z) d\vec z &= \int q_\vv(\vec
             z)\nabla_{\vec\nu} \log q_\vv(\vec z) f(\vec z)d\vec z \\
    &= \E_q[\nabla_{\vec\nu} \log q_\vv(\vec z) f(\vec z)]
  \end{align*}
}
\vspace{-5mm}
\onslide+<4->{
    \item If we can sample from $q$, this expectation can be evaluated
      easily (Monte Carlo estimation)
    \end{itemize}
    }
\end{frame}






\begin{frame}{Gradients of Expectations: Approach 1}
  $$
  \text{ELBO} = \mathcal F(\vec\nu) = \E_q[h(\vec z,\vec\nu)]\,,\quad
  h(\vec z, \vec\nu) = \log p(\vec x, \vec z) - \log q(\vec
     z|\vec\nu)
     $$
     \vspace{-5mm}
  \begin{itemize}
    \item Need gradient of ELBO w.r.t. variational parameters $\vec\nu$
  \end{itemize}
\onslide+<2->{
  \begin{align*}
    \nabla_{\vec\nu}&\mathcal F =\blue{\nabla_{\vec\nu} \E_q[h(\vec
                      z,\vec\nu)]}
                      \onslide<3->{
                      = \nabla_{\vec\nu}\int h(\vec z, \vec\nu)
                      q_\vv(\vec z)d\vec z\\}
    \onslide<4->{
                    &= \int \big(\nabla_{\vec\nu} h(\vec\nu, \vec z)\big) q_\vv(\vec z) +
                      \orange{h(\vec\nu, \vec z)\nabla_{\vec\nu}q_\vv(\vec z)}d\vec z
                      \qquad\hspace{9mm}\fbox{product rule}\\
    }
    \onslide<5->{
                    &= \int 
                      q_\vv(\vec z)\nabla_{\vec\nu}h(\vec z,
                      \vec\nu) + \orange{q_\vv(\vec z) \nabla_{\vec\nu}\log q_\vv(\vec z)
                      h(\vec z, \vec \nu)}  d\vec
                      z\quad\fbox{log-deriv. trick}\\
    }
                    &= \blue{\E_q[\nabla_{\vec\nu} \log q(\vec z|\vec \nu)h(\vec z,\vec\nu)
                      + \nabla_{\vec\nu}h(\vec z,\vec\nu)]}
  \end{align*}
}
\vspace{-6mm}
  \onslide+<6->{
  \begin{itemize}
    \item We successfully \cemph{swapped gradient and expectation}
    \item $q$ known\newline
      \arrow Sample from $q$ and use Monte Carlo estimation
    \end{itemize}
  }
\end{frame}



\begin{frame}{Score Function}
  \begin{itemize}[<+->]
  \item \cemph{Score function:} Derivative of a log-likelihood
    with respect to the parameter vector $\vec\nu$:
    \begin{myblock}{Score Function}
      \vspace{-5mm}
    \begin{align*}
      \text{score} = \nabla_{\vec\nu} \log q_\vv(\vec z) =
      \frac{1}{q_\vv(\vec z)}\nabla_{\vec\nu}q_\vv(\vec z)
    \end{align*}
  \end{myblock}
\item Measures the sensitivity of the log-likelihood
    w.r.t. $\vec\nu$
  % \item Central to maximum likelihood estimation
  \end{itemize}
  
\end{frame}



%%%%%%%%%%%%%%%%%%%%%%%%%%%%%%%%%
\begin{frame}{Score Function (2)}
  \begin{align*}
  \text{score} = \nabla_{\vec\nu} \log q_\vv(\vec z) =
  \frac{1}{q_\vv(\vec z)}\nabla_{\vec\nu}q_\vv(\vec z)
  \end{align*}
  
  \begin{itemize}
  \item Important property:
    \begin{align*}
      \E_{q_\vv(\vec z)}[\text{score}] &= \onslide+<2->{\E_{q_\vv(\vec z)}
                                  \left[\frac{1}{q_\vv(\vec z)}\nabla_{\vec\nu}q_\vv(\vec z)\right] \\
                                &= \int
                                  \frac{1}{q_\vv(\vec z)} q_\vv(\vec z)\nabla_{\vec\nu}q_\vv(\vec z) d\vec z\\
                                &= \int \nabla_{\vec\nu}q_\vv(\vec z)d\vec z =
                                  \nabla_{\vec\nu} \int q_\vv(\vec z) d\vec z = \nabla_{\vec
                                  \nu} 1 = 0}
    \end{align*}
    \onslide+<2->{\arrow \cemph{Mean of the score function is $0$}}
    \pause
  % \item Variance of the score: \cemph{Fisher information} \arrow
  \end{itemize}
\end{frame}




\begin{frame}{Score Function Gradient Estimator}
  $$\text{ELBO} = \E_q[h(\vec z,\vec\nu)] =
  \E_q[\log p(\vec x, \vec z) - \log q_\vv(\vec z)]$$
  \vspace{-6mm}
  \begin{itemize}
    \item Gradient of ELBO:
  \begin{align*}
    \nabla_{\vec\nu} \text{ELBO} &= \onslide+<2->{\E_q[\nabla_{\vec\nu} \log q_\vv(\vec z)h(\vec z,\vec\nu)] +
    \E_q[\orange{\nabla_{\vec\nu} h(\vec z,\vec\nu)}]\\}
\onslide+<3->{    &= \E_q[\nabla_{\vec\nu} \log q_\vv(\vec z)h(\vec z,\vec\nu)]\\
    &\quad + \E_q[\orange{\underbrace{\nabla_{\vec\nu}\log p(\vec x, \vec
      z)}_{=0} - \underbrace{\nabla_{\vec\nu} \log q_\vv(\vec 
      z)}_{\text{score}}}]}
  \end{align*}
  \onslide+<4->{
    \vspace{-3mm}
\item Exploit that the mean of the score function is $0$. Then:
  \begin{align*}
    \nabla_{\vec\nu} \text{ELBO} & = \E_q[\nabla_{\vec\nu} \log q_\vv(\vec z)h(\vec z,\vec\nu)] \\
    &= \E_q[\nabla_{\vec\nu}\log q_\vv(\vec z)(\log p(\vec x, \vec z) -
      \log q_\vv(\vec z)) ]
  \end{align*}
}%
\vspace{-6mm}
\onslide+<5->{%
  \item \cemph{Likelihood ratio gradient} (Glynn, 1990)
  \item \cemph{REINFORCE gradient} (Williams, 1992)%\nocite{Williams1992}
  }
\end{itemize}
\end{frame}

  
% \end{frame}
%%%%%%%%%%%%%%%%%%%%%%%%%%%%%%%%%
\begin{frame}{Using Noisy Stochastic Gradients}
  \begin{itemize}[<+->]
    \item Gradient of the ELBO
  \begin{align*}
    \nabla_{\vec\nu}\text{ELBO} = \E_q[\nabla_{\vec\nu}\log q_\vv(\vec z) (\log p(\vec x, \vec z) - \log q_\vv(\vec z))]
  \end{align*}
  is an expectation
  \item Require that $q_\vv(\vec z)$ is differentiable
    w.r.t. $\vec\nu$
\item Get noisy unbiased gradients using Monte Carlo by sampling from $q$:
  \begin{align*}
    \frac{1}{S}\sum_{s=1}^S \nabla_{\vec\nu}\log q_\vv(\vec z\idx{s}) (\log p(\vec x, \vec z\idx{s}) - \log q_\vv(\vec z\idx{s}))\,,\quad \vec z\idx{s} \sim q_\vv(\vec z)
  \end{align*}
\item Sampling from $q$ is easy (we choose $q$)
\item Use this within SVI to converge to a local optimum
\end{itemize}
\end{frame}









\begin{frame}{Summary: BBVI procedure}
Black Box Variational Inference

  \begin{enumerate}
    \item Input: model $p(\vec x, \vec z)$, variational approximation
      $q_\vv(\vec z)$
    \item Repeat
      \begin{enumerate}
      \item Draw $S$ samples $\vec z\idx{s}\sim q_\vv(\vec z)$
      \item Update variational parameters
        $$
        \vec\nu_{t+1} = \vec\nu_t + \rho_t \underbrace{\orange{\frac{1}{S}\sum_{s=1}^S \nabla_{\vec\nu}\log q(\vec
    z\idx{s}|\vec\nu) (\log p(\vec x, \vec z^{\idx{s}}) - \log q(\vec
    z\idx{s}|\vec\nu))}}_{\text{MC estimate of the score-function gradient of the ELBO}}
    $$
    \item $t=t+1$
    \end{enumerate}
  \end{enumerate}
\end{frame}




\begin{frame}{Requirements for Inference}
Similar to MCMC in that it makes \emph{few} requirements
  \begin{itemize}
  \item Computing the noisy gradient of the ELBO requires:
    \begin{itemize}
    \item Sampling from $q$. We choose $q$ so that this is possible.
    \item Evaluate the score function
      $\nabla_{\vec\nu}\log q_\vv(\vec z)$
    \item Evaluate $\log q_\vv(\vec z)$ and
      $\log p(\vec x, \vec z) = \log p(\vec z) + \log p(\vec x|\vec
      z)$
    \end{itemize}
    \arrow \cemph{No model-specific computations for optimization}
    (computations are only specific to the choice of the variational approximation)
  \end{itemize}

\end{frame}




\begin{frame}{Issue: Variance of the Gradients}
  \begin{itemize}
    \item Stochastic optimization \arrow \alert{Gradients are noisy (high variance)}
    \item The noisier the gradients, the slower the convergence
    \item Possible solutions:
      \begin{itemize}
      \item \cemph{Control variates} (with the score function as control variate)
      \item \cemph{Rao-Blackwellization}
      \item \cemph{Importance sampling}
      \end{itemize}
    \end{itemize}

\end{frame}



\begin{frame}{Issues with score function estimator}
  We can simplify the gradient estimator further:
  \begin{itemize}
    \item Score-function gradient estimator only requires general
      assumptions
    \item Noisy gradients are a problem
    \item Address this issue by making some additional assumptions
      (not too strict) \newline
      \arrow \cemph{Pathwise gradient estimators}
  \end{itemize}
\end{frame}


\begin{frame}[t]{Approach}
\begin{align}
g(\vv) = \nabla_\vv\Exp{q}{h(\vz, \vv)}
\end{align}
% \begin{center}
%     \tikzstyle{ipe stylesheet} = [
  ipe import,
  even odd rule,
  line join=round,
  line cap=butt,
  ipe pen normal/.style={line width=0.4},
  ipe pen heavier/.style={line width=0.8},
  ipe pen fat/.style={line width=1.2},
  ipe pen ultrafat/.style={line width=2},
  ipe pen normal,
  ipe mark normal/.style={ipe mark scale=3},
  ipe mark large/.style={ipe mark scale=5},
  ipe mark small/.style={ipe mark scale=2},
  ipe mark tiny/.style={ipe mark scale=1.1},
  ipe mark normal,
  /pgf/arrow keys/.cd,
  ipe arrow normal/.style={scale=7},
  ipe arrow large/.style={scale=10},
  ipe arrow small/.style={scale=5},
  ipe arrow tiny/.style={scale=3},
  ipe arrow normal,
  /tikz/.cd,
  ipe arrows, % update arrows
  <->/.tip = ipe normal,
  ipe dash normal/.style={dash pattern=},
  ipe dash dashed/.style={dash pattern=on 4bp off 4bp},
  ipe dash dotted/.style={dash pattern=on 1bp off 3bp},
  ipe dash dash dotted/.style={dash pattern=on 4bp off 2bp on 1bp off 2bp},
  ipe dash dash dot dotted/.style={dash pattern=on 4bp off 2bp on 1bp off 2bp on 1bp off 2bp},
  ipe dash normal,
  ipe node/.append style={font=\normalsize},
  ipe stretch normal/.style={ipe node stretch=1},
  ipe stretch normal,
  ipe opacity 10/.style={opacity=0.1},
  ipe opacity 30/.style={opacity=0.3},
  ipe opacity 50/.style={opacity=0.5},
  ipe opacity 75/.style={opacity=0.75},
  ipe opacity opaque/.style={opacity=1},
  ipe opacity opaque,
]
\definecolor{red}{rgb}{1,0,0}%
\begin{tikzpicture}[ipe stylesheet]
  \draw[shift={(311.397, 686.186)}, xscale=0.2262, yscale=0.0959]
    (0, 0)
     .. controls (10.6667, 26.6667) and (58.6667, 74.6667) .. (109.3333, 69.3333)
     .. controls (160, 64) and (213.3333, 5.3333) .. (234.6667, -50.6667)
     .. controls (256, -106.6667) and (245.3333, -160) .. (208, -200)
     .. controls (170.6667, -240) and (106.6667, -266.6667) .. (88, -248)
     .. controls (69.3333, -229.3333) and (96, -165.3333) .. (96, -122.6667)
     .. controls (96, -80) and (69.3333, -58.6667) .. (42.6667, -45.3333)
     .. controls (16, -32) and (-10.6667, -26.6667) .. cycle;
  \pic
     at (339.3985, 680.6362) {ipe disk};
  \draw[shift={(339.398, 680.636)}, scale=0.1686, -{>[ipe arrow small]}]
    (0, 0)
     .. controls (48, 16) and (112, -48) .. (158.005, -9.169);
  \pic
     at (366.0336, 679.0905) {ipe disk};
  \node[ipe node]
     at (84, 708) {$p(\vec x, \vec z)$};
  \node[ipe node]
     at (84, 648) {$q(\vec z|\vec \nu)$};
  \draw[shift={(196, 696)}, xscale=1.25,  ipe pen fat]
    (0, 0) rectangle (64, -32);
  \node[ipe node]
     at (204.891, 677.511) {$\displaystyle \int (...) q(\vec z|\vec
       \nu) d\vec z$};
  \node[ipe node]
     at (141.371, 677.339) {$\nabla_{\vec \nu}$};
  \draw[ipe pen fat]
    (132, 696) rectangle (164, 664);
  \draw[->]
    (164, 680)
     -- (196, 680);
  \draw[->]
    (276, 680)
     -- (308, 680);
  \draw[->]
    (112, 656)
     -- (132, 664);
  \draw[->]
    (112, 704)
     -- (132, 696);
  \node[ipe node, font=\tiny]
     at (245, 646.47) {\it \sffamily Adopted from Blei et al.'s NIPS-2016 tutorial};
\end{tikzpicture}

%%% Local Variables:
%%% mode: latex
%%% TeX-master: "../lecture_variational_inference"
%%% End:

%   \end{center}
  \begin{itemize}[<+->]
    \item Switch order to integration first, then differentiation
    \item Write integration as expectation again
    \item Approximate the expectation after having taken the gradient\newline
      \arrow Monte Carlo estimator (ideally with low variance)
    \item Stochastic optimization
    \end{itemize}
    \onslide+<5->{
    \arrow Require: \cemph{general way to compute gradients of
      expectations}}
\end{frame}



\begin{frame}{Change of Variables}
Some distributions can be sampled using a \emph{change of variables}, i.e.
\begin{gather*}
\vz = \vec t(\vepsilon) \qquad \text{with } \vepsilon \sim p(\vepsilon)
\implies p(\vz) \text{ some desired distribution}
\end{gather*}
Densities are related
\begin{gather*}
p(\vepsilon) = p(\vz = t(\vepsilon)) \pderiv[t(\vepsilon)]{\vepsilon}
\end{gather*}
Integrals are related
\begin{gather*}
\int h(\vz) p(\vz) \calcd{\vz} = \int h(t(\vepsilon)) p(\vz = t(\vepsilon)) \pderiv[t(\vepsilon)]{\vepsilon} \calcd{\vepsilon} = \int h(t(\vepsilon)) p(\vepsilon) \calcd{\vepsilon}
\end{gather*}
  \begin{center}
  \tikzstyle{ipe stylesheet} = [
  ipe import,
  even odd rule,
  line join=round,
  line cap=butt,
  ipe pen normal/.style={line width=0.4},
  ipe pen heavier/.style={line width=0.8},
  ipe pen fat/.style={line width=1.2},
  ipe pen ultrafat/.style={line width=2},
  ipe pen normal,
  ipe mark normal/.style={ipe mark scale=3},
  ipe mark large/.style={ipe mark scale=5},
  ipe mark small/.style={ipe mark scale=2},
  ipe mark tiny/.style={ipe mark scale=1.1},
  ipe mark normal,
  /pgf/arrow keys/.cd,
  ipe arrow normal/.style={scale=7},
  ipe arrow large/.style={scale=10},
  ipe arrow small/.style={scale=5},
  ipe arrow tiny/.style={scale=3},
  ipe arrow normal,
  /tikz/.cd,
  ipe arrows, % update arrows
  <->/.tip = ipe normal,
  ipe dash normal/.style={dash pattern=},
  ipe dash dashed/.style={dash pattern=on 4bp off 4bp},
  ipe dash dotted/.style={dash pattern=on 1bp off 3bp},
  ipe dash dash dotted/.style={dash pattern=on 4bp off 2bp on 1bp off 2bp},
  ipe dash dash dot dotted/.style={dash pattern=on 4bp off 2bp on 1bp off 2bp on 1bp off 2bp},
  ipe dash normal,
  ipe node/.append style={font=\normalsize},
  ipe stretch normal/.style={ipe node stretch=1},
  ipe stretch normal,
  ipe opacity 10/.style={opacity=0.1},
  ipe opacity 30/.style={opacity=0.3},
  ipe opacity 50/.style={opacity=0.5},
  ipe opacity 75/.style={opacity=0.75},
  ipe opacity opaque/.style={opacity=1},
  ipe opacity opaque,
]
\definecolor{red}{rgb}{1,0,0}%
\definecolor{green}{rgb}{0,1,0}%
\definecolor{blue}{rgb}{0,0,1}%
\definecolor{yellow}{rgb}{1,1,0}%
\definecolor{orange}{rgb}{1,0.647,0}%
\definecolor{gold}{rgb}{1,0.843,0}%
\definecolor{purple}{rgb}{0.627,0.125,0.941}%
\definecolor{gray}{rgb}{0.745,0.745,0.745}%
\definecolor{brown}{rgb}{0.647,0.165,0.165}%
\definecolor{navy}{rgb}{0,0,0.502}%
\definecolor{pink}{rgb}{1,0.753,0.796}%
\definecolor{seagreen}{rgb}{0.18,0.545,0.341}%
\definecolor{turquoise}{rgb}{0.251,0.878,0.816}%
\definecolor{violet}{rgb}{0.933,0.51,0.933}%
\definecolor{darkblue}{rgb}{0,0,0.545}%
\definecolor{darkcyan}{rgb}{0,0.545,0.545}%
\definecolor{darkgray}{rgb}{0.663,0.663,0.663}%
\definecolor{darkgreen}{rgb}{0,0.392,0}%
\definecolor{darkmagenta}{rgb}{0.545,0,0.545}%
\definecolor{darkorange}{rgb}{1,0.549,0}%
\definecolor{darkred}{rgb}{0.545,0,0}%
\definecolor{lightblue}{rgb}{0.678,0.847,0.902}%
\definecolor{lightcyan}{rgb}{0.878,1,1}%
\definecolor{lightgray}{rgb}{0.827,0.827,0.827}%
\definecolor{lightgreen}{rgb}{0.565,0.933,0.565}%
\definecolor{lightyellow}{rgb}{1,1,0.878}%
\definecolor{black}{rgb}{0,0,0}%
\definecolor{white}{rgb}{1,1,1}%
%%% Local Variables:
%%% mode: latex
%%% TeX-master: t
%%% End:

\begin{tikzpicture}[ipe stylesheet]
  \fill[blue!40!white]
    (128, 736) rectangle (160, 704);
  \fill[orange!40!white]
    (320, 704)
     -- (256, 736)
     -- (288, 768)
     -- (352, 736)
     -- (320, 704);
  \draw[ipe pen ultrafat, -{>[ipe arrow small]}]
    (320, 704)
     -- (256, 736);
  \draw[ipe pen ultrafat, -{>[ipe arrow small]}]
    (320, 704)
     -- (352, 736);
  \draw[ipe pen ultrafat, -{>[ipe arrow small]}]
    (128, 704)
     -- (160, 704);
  \draw[ipe pen ultrafat, -{>[ipe arrow small]}]
    (128, 704)
     -- (128, 736);
  \node[ipe node]
     at (140, 692) {$\vec b_1$};
  \node[ipe node]
     at (112, 720) {$\vec b_2$};
  \node[ipe node]
     at (260, 720) {$\vec c_1$};
  \node[ipe node]
     at (348, 720) {$\vec c_2$};
  \draw[ipe pen heavier, -{>[ipe arrow small]}]
    (176, 736)
     arc[start angle=-129.0939, end angle=-75.9638, x radius=82.4621, y radius=-82.4621];
  \node[ipe node]
     at (204, 736) {$\vec t(\cdot)$};
\end{tikzpicture}

%%% Local Variables:
%%% mode: latex
%%% TeX-master: "../mml-book"
%%% End:

\end{center}
% \begin{itemize}
%   \item Use function $\vec
%     \vepsilon\mapsto \vec t(\vec \epsilon) = \vec \vz$
%   \item Gives new density
%     \begin{align}
%       p(\vz) = p(\vepsilon = \vec t\inv(\vz))
%     \end{align}
%   \end{itemize}
\end{frame}




\begin{frame}{Gradients of Expectations: Approach 2}
  \begin{align*}
    \nabla_{\vec\nu}\text{ELBO} &= \blue{\nabla_{\vec\nu}\E_q[g(\vec
                                  z, \vec\nu)]} \\
                                \onslide+<2->{&= \nabla_{\vec\nu}\int
                                  g(\vec z, \vec\nu) q_\vv(\vec z) d\vec z\\}
                                \onslide+<3->{&=\nabla_{\vec\nu}\int g(\vec z, \vec\nu)
                                  q(\vec\epsilon)d\vec\epsilon \qquad\qquad
                                  \fbox{$q(\vec z) d\vec z = q(\vec\epsilon)d\vec\epsilon$}\\}
                                \onslide+<4->{&= \nabla_{\vec\nu}\int g(t(\vec
                                  \epsilon, \vec\nu),
                                  \vec\nu)q(\vec\epsilon)d\vec\epsilon\qquad\fbox{$\vec z =
                                  t(\vec \epsilon, \vec\nu)$}\\}
                                \onslide+<5->{&= \int \nabla_{\vec\nu} g(t(\vec\epsilon,
                                  \vec\nu), \vec\nu){q(\vec\epsilon)}d\vec\epsilon\qquad
                                  \fbox{$\nabla_{\vec\nu}\int_{\vec\epsilon}
                                  = \int_{\vec\epsilon}\nabla_{\vec\nu}$}\\}
                                \onslide+<6->{&= \blue{\E_{q(\vec\epsilon)}[\nabla_{\vec\nu}g( t(\vec\epsilon,\vec\nu), \vec\nu)]}}
  \end{align*}
\onslide+<7->{\arrow Turned gradient of an expectation into
  expectation of a gradient (and sampling from $q(\vec\epsilon)$ is
  very easy).}
\end{frame}






\begin{frame}{Reparametrization Trick}


  \begin{myblock2}{Reparametrization Trick}
    Base distribution $p(\vec\epsilon)$ and a deterministic
    transformation $\vec z = t(\vec\epsilon, \vec\nu)$ so that $ \vec
    z \sim q_\vv(\vec z)$. Then:
    \begin{align*}
      \nabla_{\vec\nu}\E_{q_\vv(\vec z)}[f(\blue{\vec z})] =
      \E_{p(\vec\epsilon)}[\nabla_{\vec\nu}f(\blue{t(\vec\epsilon, \vec\nu)})]
    \end{align*}
    \arrow Expectation taken w.r.t. base distribution
  %  \arrow We can swap differentiation and integration
  \end{myblock2}
\pause
  \begin{itemize}
    \item Key idea: change of variables using a deterministic transformation
  \end{itemize}
\end{frame}

%%%%%%%%%%%%%%%%%%%%%%% 
\begin{frame}{Example}
  \begin{figure}
    \includegraphics[width=0.4\hsize]{./figures/vi/standard_normal}
    \hspace{2mm}
    \includegraphics[width=0.4\hsize]{./figures/vi/reparam_trick_example}
  \end{figure}
  \begin{align*}
    \vec\nu &:=\{\vec\mu, \mat R\}\,,\quad \mat R\mat R\T = \mat\Sigma\\
    p(\vec\epsilon) &= \gauss{\vec 0}{\mat I}\\
    t(\vec\epsilon, \vec\nu) &= \vec\mu + \mat R\vec\epsilon\\
    \implies p(\vec z) &= \gaussx{\vec z}{\vec\mu}{\mat\Sigma}
  \end{align*}
  \begin{itemize}
  \item Base distribution is parameter free
  \item Simple deterministic transformation $t$ generate more
    expressive distributions\footnote{\url{https://tinyurl.com/hyakoj2}}
  \end{itemize}
  
\end{frame}



\begin{frame}{Pathwise Gradients}
\begin{align*}
    g(\vec z , \vec \nu) &= \log p(\vec x, \vec z) - \log q(\vec
    z|\vec\nu)\\
  \vec z &= t(\vec \epsilon, \vec\nu)
  \end{align*}
  %
  Simplify gradient of the ELBO:
  \pause
  \begin{align*}
    \nabla_{\vec\nu}&\text{ELBO}= \E_{p(\vec\epsilon)}[\nabla_{\vec\nu} g(t(\vec\epsilon, \vec\nu),
                                 \vec\nu)]\\
                               &=
                                 \E_{p(\vec\epsilon)}[\blue{\nabla_{\vec\nu} \log  p(\vec x,
                                 t(\vec\epsilon,\vec\nu))} -
                                 \orange{\nabla_{\vec\nu} \log
                                 q(t(\vec\epsilon,\vec\nu)|\vec\nu)}]\quad\fbox{Def. of
                                 $g$}\\
                               &= \E_{p(\vec\epsilon)}[\blue{\nabla_{\vec z}\log p(\vec x, \vec
                                 z) \nabla_{\vec\nu}
                                 t(\vec\epsilon,\vec\nu
                                 )}\\
                               &\quad -
                                 \orange{\nabla_{\vec z} \log q(\vec
                                 z|\vec\nu)\nabla_{\vec \nu}t(\vec \epsilon,\vec\nu)
                                 -\underbrace{\nabla_{\vec\nu}\log q(t(\vec\epsilon,\vec\nu)|\vec\nu)}_{\text{score}}}]\quad\fbox{Chain rule}\\
                               &=\E_{p(\vec\epsilon)}[\nabla_{\vec z}\big(\log p(\vec x, \vec z) -\log
                                 q_\vv(\vec z)\big)\nabla_{\vec\nu}
                                 t(\vec\epsilon, \vec\nu)]\quad\hspace{3mm} \fbox{Score property}
  \end{align*}
  \vspace{-4mm}
  \begin{itemize}
  \item \cemph{Pathwise gradient}
  \item \cemph{Reparametrization gradient}
  \end{itemize}
\end{frame}



\begin{frame}{Variance Comparison}
  \begin{figure}
    \centering
    \includegraphics[height=3cm]{./figures/vi/var_comp_score_rep}
  \end{figure}
  \begin{itemize}[<+->]
    \item Drastically reduced variance compared to score-function
      gradient estimation
    \item Restricted class of models (compared with score function
      estimator)
  \end{itemize}
\end{frame}



\begin{frame}{Score Function vs Pathwise
    Gradients}
  \begin{myblock}{}
    \vspace{-5mm}
  \begin{align*}
    \text{ELBO} &= \int g(\vec z, \vec\nu) q_\vv(\vec z)d\vec z\\
    g(\vec z, \vec\nu) &= \log p(\vec x, \vec z) - \log q(\vec z|\vec\mu)
  \end{align*}
\end{myblock}
\begin{itemize}
  \item Score function gradient:
    $$
\nabla_{\vec\nu}\text{ELBO} = \E_q[\blue{\big(\nabla_{\vec\nu}\log q(\vec
z|\vec\nu)\big)} g(\vec z, \vec\nu)]
$$
\arrow Gradient of the variational distribution
  \item Reparametrization gradient:
    $$
\nabla_{\vec\nu}\text{ELBO} =\E_{p(\vec\epsilon)}[\blue{\big(\nabla_{\vec
    z}g(\vec z,\vec\nu)\big)}\nabla_{\vec\nu}
      t(\vec\epsilon, \vec\nu)]
      $$
      \arrow Gradient of the model and the variational distribution
    \item Often, $\Exp{q_\vv(\vz)}{\log q_\vv(\vz)}$ can be computed in closed form, and is excluded from MC estimation. (Skill to recognise when.)
  \end{itemize}
  
\end{frame}


\begin{frame}{Summary}

  \begin{itemize}
  \item Score function
    \begin{itemize}
    \item \cemph{Works for all models (continuous and discrete)}
    \item \cemph{Works for a large class of variational approximations}
    \item \calert{Variance} can be high \arrow Slow convergence
    \end{itemize}
  \item Pathwise gradient estimator
    \begin{itemize}
    \item Requires \calert{differentiable models}
    \item Requires the \calert{variational approximation to be
        expressed as a deterministic transformation
        $\vec z=t(\vec\epsilon, \vec\nu)$}
    \item \blue{Generally lower variance}
    \end{itemize}
  \end{itemize}
\end{frame}










%%%%%%%%%%%%%%%%%%%%%%%%%%%%%%%%%%%%%%%%%
% REFERENCES
%%%%%%%%%%%%%%%%%%%%%%%%%%%%%%%%%%%%%%%%%
\begin{frame}[t,allowframebreaks]
\frametitle{References}
\linespread{1.0}
\tiny
\bibliographystyle{abbrv}
\bibliography{../includes/pi-literature}
\end{frame}



\end{document}
%%% Local Variables: 
%%% mode: latex
%%% TeX-master: t
%%% End: 
